\documentclass[tikz, border=10pt]{standalone}
\usepackage{tikz}
\usetikzlibrary{positioning, fit, shapes.geometric, calc}

% Define styles for nodes
\tikzset{
  process/.style={rectangle, draw, minimum width=3cm, minimum height=1cm},
  database/.style={cylinder, shape border rotate=90, aspect=0.25, draw},
  arrow/.style={->, thick, >=stealth, draw},
}

% Define a custom command to create vertically aligned nodes
\newcommand{\createVerticalNodes}[4]{%
  % #1: Name prefix for nodes
  % #2: List of node labels
  % #3: Starting point
  % #4: Node style
  \coordinate (previous) at #3;
  \foreach \label [count=\i from 1] in {#2} {
    \ifnum\i=1
      \node (#1\i) [#4, below=0 of previous] {\label};
    \else
      \node (#1\i) [#4, below=of #1\the\numexpr\i-1\relax] {\label};
    \fi
  }
}

\begin{document}
\begin{tikzpicture}[node distance=0.4cm]

% Satellite Image Database and its internals
\node (satelliteImageInternals) [process] {Satellite Image (03)};
\node (satelliteImageMetaData) [process, below=of satelliteImageInternals] {Satellite Image Meta Data (04)};

\node (satelliteImage) [process, fit={(satelliteImageInternals) (satelliteImageMetaData)}, inner sep=0.5cm, label=above:{\textbf{Satellite Image (02)}}] {};

\node (satelliteDb) [database, fit={(satelliteImage)}, inner sep=0.5cm, label=above:{\textbf{Satellite Image Database (05)}}] {};








% Dwelling Detector
% Starting point to the right of satelliteDb

\coordinate (inputProcessorStart) at ($(satelliteDb.east) + (4cm,3cm)$);

\createVerticalNodes{inputProcessorNode}
    {Training Data (02A), Validation Data (02B)}
    {(inputProcessorStart)}
    {process}

\node (inputProcessor) [process, fit=(inputProcessorNode1) (inputProcessorNode2), label=above:\textbf{Input Processor (10)}] {};


\coordinate (modelGeneratorStart) at ($(inputProcessor.south) + (0cm, -1cm)$);

\node (model) [process, below=0 of modelGeneratorStart] {Model (12)};

\node (modelGenerator) [process, fit=(model), label=above:\textbf{Model Generator (11)}] {};


\coordinate (dwellingStart) at ($(inputProcessor.east)+(2cm,1cm)$);

% Use the custom command to create vertical nodes within the Dwelling Detector
\createVerticalNodes{dwellingNode}
  {Output Processor (14), Temporal Analyzer (54), Clustering Process (16)}
  {(dwellingStart)}
  {process}

% Fit a node around the created nodes to represent the Dwelling Detector
\node (dwellingDetector) [process, fit=(inputProcessor) (modelGenerator) (dwellingNode3), inner sep=0.5cm, label=above:{\textbf{Dwelling Detector (01)}}] {};



% Model Database and its internals
\coordinate (modelDbStart) at ($(dwellingDetector.east) + (4cm, 4cm)$);

\createVerticalNodes{modelDbNode}
    {Parameters (21), Weights (22)}
    {(modelDbStart)}
    {process}

\node (modelDb) [database, fit=(modelDbNode1) (modelDbNode2), inner sep=0.5cm, label=above:{\textbf{Model Database (20)}}] {};



% Dwelling Database and its internals
\coordinate (dwellingDbStart) at ($(modelDb.south) + (0cm, -3cm)$);

\createVerticalNodes{dwellingDbNode}
    {Dwelling Location (41), Clustering of Dwellings (45)}
    {(dwellingDbStart)}
    {process}

\node (dwellingDb) [database, fit=(dwellingDbNode1) (dwellingDbNode2), inner sep=0.5cm, label=above:{\textbf{Dwelling Database (40)}}] {};


% Dwelling Metadata
\coordinate (dwMetadataStart) at ($(modelDb.east) + (4cm, -3.5cm)$);

\createVerticalNodes{metadataNode}
    {Geographical Location (42), Time Stamp (43), Resource Instruction (44)}
    {(dwMetadataStart)}
    {process}

\node (metadataContainer) [process, fit=(metadataNode1) (metadataNode3), inner sep=0.5cm] {};



% Resource Provisioning Unit and its internals

\coordinate (resourceStart) at ($(modelDb.east) + (4cm, 2cm)$);

\createVerticalNodes{resourceNode}
    {Water Consumption Estimation (51), Water Unit Calculator (52)}
    {(resourceStart)}
    {process}

\node (resourceProvisioningUnit) [process, fit=(resourceNode1) (resourceNode2), inner sep=0.5cm, label=above:{\textbf{Resource Provisioning Unit (50)}}] {};






% Connectors
\draw [arrow] (satelliteDb) -- (inputProcessor);
\draw [arrow] (inputProcessor) -- (modelGenerator);
\draw [arrow] (model) -- (dwellingNode1);
\draw [arrow] (model) -- (modelDb);
\draw [arrow] (modelDb) -- (model);
\draw [arrow] (dwellingNode1) -- (dwellingDbNode1);
\draw [arrow] (dwellingNode3) -- (dwellingDbNode2);
\draw [arrow] (dwellingDbNode2) -- (dwellingNode3);
\draw [arrow] (dwellingDbNode1) -- (metadataNode1);
\draw [arrow] (dwellingDbNode1) -- (metadataNode2);
\draw [arrow] (dwellingDbNode1) -- (metadataNode3);
\draw [arrow] (dwellingDb) -- (resourceProvisioningUnit)

\end{tikzpicture}

\end{document}